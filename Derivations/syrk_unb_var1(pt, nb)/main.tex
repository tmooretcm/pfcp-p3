\documentclass[12pt]{article}

\usepackage{amssymb}
\usepackage{ifthen}
\usepackage[table]{xcolor}
\usepackage{minitoc}
\usepackage{array}

\definecolor{yellow}{cmyk}{0,0,1,0}
\renewcommand{\arraystretch}{1.4}
\newcommand{\R}{\mathbb{R}}

\usepackage{colortbl}

% Page size
\setlength{\oddsidemargin}{-0.5in}
\setlength{\evensidemargin}{-0.5in}
\setlength{\textheight}{10.25in}
\setlength{\textwidth}{7.0in}
\setlength{\topmargin}{-1.35in}

\renewcommand{\arraycolsep}{3pt}

\pagenumbering{gobble}

\input color_flatex

\begin{document}
\pagestyle{empty}
\resetsteps % reset all definitions






\resetsteps      % Reset all the commands to create a blank worksheet  

% Define the operation to be computed

\renewcommand{\operation}{ \left[ C \right] := \mbox{\sc syrk\_unb\_var1}( A, C) }

\renewcommand{\routinename}{\operation}

% Step 1a: Precondition 

\renewcommand{\precondition}{
  C = \widehat{C}
}

% Step 1b: Postcondition 

\renewcommand{\postcondition}{ 
  \left[C \right]
  =
  \mbox{syrk}( A, \widehat{C})
}

% Step 2: Invariant 
% Note: Right-hand side of equalities must be updated appropriately

\renewcommand{\invariant}{
  C = A_{T}^TA_{T} + \widehat C
}

% Step 3: Loop-guard 

\renewcommand{\guard}{
  m( A_T ) < m( A )
}

% Step 4: Initialize 

\renewcommand{\partitionings}{
  $
  A \rightarrow
  \left(\begin{array}{c}
     A_{T} \\ \whline 
     A_{B}
  \end{array}\right)
  $
,
  $
  A^T \rightarrow
  \left(\begin{array}{c I c}
     A^T_T & A^T_B
  \end{array}\right)
  $
}

\renewcommand{\partitionsizes}{
  $ A_T $ has $ 0 $ rows,
  $ A^T_T $ has $ 0 $ columns
}

% Step 5a: Repartition the operands 

\renewcommand{\repartitionings}{
  $  \left(\begin{array}{c}
     A_T \\ \whline
     A_B 
  \end{array}\right) 
  \rightarrow
  \left(\begin{array}{c}
     A_0 \\ \whline 
     a_1^T \\  
     A_2
  \end{array}\right)
  $
% ,
%   $  \left(\begin{array}{c I c}
%      B_L & B_R
%   \end{array}\right)
%   \rightarrow
%   \left(\begin{array}{c I c c}
%      B_0 & b_1 & B_2
%   \end{array}\right)
%   $
}

\renewcommand{\repartitionsizes}{
  $ a_1 $ has $ 1 $ row
  % ,$ b_1 $ has $ 1 $ column}
  }

% Step 5b: Move the double lines 

\renewcommand{\moveboundaries}{
$  \left(\begin{array}{c}
     A_T \\ \whline
     A_B 
  \end{array}\right) 
  \leftarrow
  \left(\begin{array}{c}
     A_0 \\  
     a_1^T \\ \whline 
     A_2
  \end{array}\right) 
  $
% ,
% $  
%   B \rightarrow
%   \left(\begin{array}{c I c}
%      B_L & B_R
%   \end{array}\right)
%   \leftarrow
%   \left(\begin{array}{c c I c}
%      B_0 & b_1 & B_2
%   \end{array}\right)
%   $
}

% Step 6: State before update
% Note: The below needs editing consistent with loop-invariant!!!

\renewcommand{\beforeupdate}{$ 
C = A_{0}^TA_{0} + \widehat C
$}


% Step 7: State after update
% Note: The below needs editing consistent with loop-invariant!!!

\renewcommand{\afterupdate}{$ 
C = A_{0}^TA_{0} + a_{1}a_{1}^T + \widehat C
$}


% Step 8: Insert the updates required to change the 
%         state from that given in Step 6 to that given in Step 7
% Note: The below needs editing!!!

\renewcommand{\update}{
$
  \begin{array}{l}          % do not delete this line 
    C := a_{1}a_{1}^T + \widehat C
  \end{array}               % do not delete this line 
$
}


\begin{center}
	\FlaWorksheet
\end{center}

\begin{figure}[p]
\begin{center}
	\FlaWorksheetZero
\end{center}
\end{figure}

\begin{figure}[p]
\begin{center}
	\FlaWorksheetOne
\end{center}
\end{figure}

\begin{figure}[p]
\begin{center}
	\FlaWorksheetTwo
\end{center}
\end{figure}

\begin{figure}[p]
\begin{center}
	\FlaWorksheetThree
\end{center}
\end{figure}

\begin{figure}[p]
	\begin{center}
	\FlaWorksheetFour
\end{center}
\end{figure}

\begin{figure}[p]
	\begin{center}
	\FlaWorksheetFive
\end{center}
\end{figure}

\begin{figure}[p]
	\begin{center}
	\FlaWorksheetSix
\end{center}
\end{figure}

\begin{figure}[p]
	\begin{center}
	\FlaWorksheetSeven
\end{center}
\end{figure}

\begin{figure}[p]
	\begin{center}
	\FlaWorksheetEight
\end{center}
\end{figure}

\begin{figure}[p]
	\begin{center}
	\FlaWorksheetNine
\end{center}
\end{figure}
%
%\begin{figure}[p]
%\begin{center}
%	\FlaAlgorithm
%\end{center}
%\end{figure}

\end{document}