\documentclass[12pt]{article}

\usepackage{amssymb}
\usepackage{ifthen}
\usepackage[table]{xcolor}
\usepackage{minitoc}
\usepackage{array}

\definecolor{yellow}{cmyk}{0,0,1,0}
\renewcommand{\arraystretch}{1.4}
\newcommand{\R}{\mathbb{R}}

\usepackage{colortbl}

% Page size
\setlength{\oddsidemargin}{-0.5in}
\setlength{\evensidemargin}{-0.5in}
\setlength{\textheight}{10.25in}
\setlength{\textwidth}{7.0in}
\setlength{\topmargin}{-1.35in}

\renewcommand{\arraycolsep}{3pt}

\pagenumbering{gobble}

\input color_flatex

\begin{document}
\pagestyle{empty}
\resetsteps % reset all definitions



\resetsteps      % Reset all the commands to create a blank worksheet  

% Define the operation to be computed

\renewcommand{\operation}{ \left[ C \right] := \mbox{\sc syrk\_unb\_var2}( A, C ) }

\renewcommand{\routinename}{\operation}

% Step 1a: Precondition 

\renewcommand{\precondition}{
  C = \widehat{C}
}

% Step 1b: Postcondition 

\renewcommand{\postcondition}{ 
   C = A^TA + \widehat{C}
}

% Step 2: Invariant 
% Note: Right-hand side of equalities must be updated appropriately

\renewcommand{\invariant}{
  \left(\begin{array}{c I c}
     C_{TL} & C_{TR} \\ \whline 
     C_{BL} & C_{BR}
  \end{array}\right) =
  \left(\begin{array}{c I c}
     A_L^TA_L + \widehat C_{TL} & * \\ \whline
     \widehat C_{BL} & \widehat C_{BR}
  \end{array}\right)
}

% Step 3: Loop-guard 

\renewcommand{\guard}{
  n( A_{L} ) < n( A )
}

% Step 4: Initialize 

\renewcommand{\partitionings}{
  $
  A \rightarrow
  \left(\begin{array}{c I c} 
     A_{L} & A_{R}
  \end{array}\right) 
  $
,
  $
  A^T \rightarrow
  \left(\begin{array}{c} 
     A_{L}^T\\ \whline
     A_{R}^T
  \end{array}\right) 
  $
,
  $
  C \rightarrow
  \left(\begin{array}{c I c} 
     C_{TL} & * \\ \whline
     C_{BL} & C_{BR} 
  \end{array}\right) 
  $
}

\renewcommand{\partitionsizes}{
  $ A $ is $ n \times n $,
  $ A^T $ is $ n \times n $,
  $ C $ is $ n \times n $
  
}

% Step 5a: Repartition the operands 

\renewcommand{\repartitionings}{
  $ A \rightarrow
  \left(\begin{array}{c I c C} 
     A_{0} & \alpha_{1} & A_2
  \end{array}\right) 
  $
,
  $
  A^T \rightarrow
  \left(\begin{array}{c} 
     A_{0}^T\\ \whline
     \alpha_{1}^T \\
     A_2^T
  \end{array}\right) 
  $
,
  $  \left(\begin{array}{c I c}
     C_{TL} & * \\ \whline 
     C_{BL} & C_{BR}
  \end{array}\right) 
  \rightarrow
  \left(\begin{array}{c I c c}
     C_{00} & * & * \\ \whline 
     c_{10}^T & \gamma_{11} & c_{12}^T \\  
     C_{20} & c_{21} & C_{22}
  \end{array}\right) 
  $
}

\renewcommand{\repartitionsizes}{
  $ a_1 $ has $ 1 $ column,
  $ \gamma_{11} $ is $ 1 \times 1 $}

% Step 5b: Move the double lines 

\renewcommand{\moveboundaries}{
$ A \leftarrow
  \left(\begin{array}{c c I c} 
     A_{0} & \alpha_{1} & A_2
  \end{array}\right) 
  $
,
  $
  A^T \leftarrow
  \left(\begin{array}{c} 
     A_{0}^T\\ 
     \alpha_{1}^T \\ \whline
     A_2^T
  \end{array}\right) 
  $
,
$  \left(\begin{array}{c I c}
     C_{TL} & * \\ \whline 
     C_{BL} & C_{BR}
  \end{array}\right) 
  \leftarrow
  \left(\begin{array}{c c I c}
     C_{00} & c_{01} & * \\  
     c_{10}^T & \gamma_{11} & *\\ \whline 
     C_{20} & c_{21} & C_{22}
  \end{array}\right) 
  $
}

% Step 6: State before update
% Note: The below needs editing consistent with loop-invariant!!!

\renewcommand{\beforeupdate}{$
\left(\begin{array}{c c c}
     C_{00} & * & *\\ 
     c_{10}^T & \gamma_{11} & c_{12}^T \\  
     C_{20} & c_{21} & C_{22}
  \end{array}\right) 
  = 
  \left(\begin{array}{c c c}
     A_0^TA_0 + C_{00} & * & * \\  
      c_{10}^T & \gamma_{11} & c_{12}^T \\  
     C_{20} & c_{21} & C_{22}
  \end{array}\right) 
$}


% Step 7: State after update
% Note: The below needs editing consistent with loop-invariant!!!

\renewcommand{\afterupdate}{$
\left(\begin{array}{c c c}
     C_{00} & c_{01} & *\\ 
     c_{10}^T & \gamma_{11} & * \\  
     C_{20} & c_{21} & C_{22}
  \end{array}\right) 
  = 
  \left(\begin{array}{c c c}
     A_0^TA_0 + C_{00} & A_0^T\alpha_1 + c_{01} & * \\  
      \alpha_1^TA_0 + c_{10}^T & \alpha_1^T\alpha_1 + \gamma_{11} & * \\  
     C_{20} & c_{21} & C_{22}
  \end{array}\right) 
$}


% Step 8: Insert the updates required to change the 
%         state from that given in Step 6 to that given in Step 7
% Note: The below needs editing!!!

\renewcommand{\update}{
$
  \begin{array}{l}          % do not delete this line 
    c_{10}^T := \alpha_1^TA_0 + \widehat{c_{10}^T}\\
    \gamma_{11} := \alpha_1^T\alpha_1 + \widehat{\gamma_{11}}\\
    c_{01} := c_{10}^T^T
  \end{array}               % do not delete this line 
$
}















\begin{center}
	\FlaWorksheet
\end{center}

\begin{figure}[p]
\begin{center}
	\FlaWorksheetZero
\end{center}
\end{figure}

\begin{figure}[p]
\begin{center}
	\FlaWorksheetOne
\end{center}
\end{figure}

\begin{figure}[p]
\begin{center}
	\FlaWorksheetTwo
\end{center}
\end{figure}

\begin{figure}[p]
\begin{center}
	\FlaWorksheetThree
\end{center}
\end{figure}

\begin{figure}[p]
	\begin{center}
	\FlaWorksheetFour
\end{center}
\end{figure}

\begin{figure}[p]
	\begin{center}
	\FlaWorksheetFive
\end{center}
\end{figure}

\begin{figure}[p]
	\begin{center}
	\FlaWorksheetSix
\end{center}
\end{figure}

\begin{figure}[p]
	\begin{center}
	\FlaWorksheetSeven
\end{center}
\end{figure}

\begin{figure}[p]
	\begin{center}
	\FlaWorksheetEight
\end{center}
\end{figure}

\begin{figure}[p]
	\begin{center}
	\FlaWorksheetNine
\end{center}
\end{figure}
%
%\begin{figure}[p]
%\begin{center}
%	\FlaAlgorithm
%\end{center}
%\end{figure}

\end{document}